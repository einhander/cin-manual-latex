% FIXME Settings
% - Page numbers always on the right, matching the chapter
%   number. Possibly coloring.
% - Improve header, remove the line, because with the listings also
%   come lines and when they meet it looks bad. Set italics and
%   color if necessary.
% - But be careful, too much ink can quickly have a negative
%   effect.

% FIXME Text
% - Change all tables to variable dimensions. -> use draft
% - (Remove all leading \$ characters from the shell
%   examples. Alternative: invalidate according to TeX rules, so that
%   other editors do not consider this an error. Do not automate, this
%   must be adjusted manually.)
% - Done. Remove "images/" from the path of includegraphics, the
%   "graphicspath" is already set.


% The Settings

% In loose reference to the theme CW.
\definecolor{CinBlueText}{RGB}{23,85,142}% "Dark blue"
\definecolor{CinBlue}{RGB}{35,134,220}%    "Light blue"
\definecolor{CinRed}{RGB}{205,38,11}%      "negativ"
\definecolor{CinOrange}{RGB}{250,125,0}%   "neutral"
\definecolor{CinGreen}{RGB}{39,174,96}%    "positiv"
\definecolor{CinSilver}{RGB}{127,140,141}%
\definecolor{CinWhite}{RGB}{239,240,241}%

% Original
% \definecolor{chaptercolour}{RGB}{23,85,142}
% \definecolor{sectioncolour}{RGB}{23,85,142}
% \definecolor{subsectioncolour}{RGB}{23,85,142}
% \definecolor{subsubsectioncolour}{RGB}{23,85,142}

\makechapterstyle{morrow}{% requires graphicx package
  \chapterstyle{default}
  \renewcommand*{\chapnamefont}{%
    \normalfont\Large\scshape\raggedleft\color{CinBlueText}}
  \renewcommand*{\chapnumfont}{%
    \normalfont\Large\bfseries\sffamily\color{CinBlueText}}
  \renewcommand*{\printchapternum}{%
    \chapnumfont \resizebox{!}{3ex}{\thechapter}}
  \renewcommand*{\afterchapternum}{% FIXME vskip?
    \par\hspace{1.5cm}\hrule\vskip\midchapskip}
  \renewcommand*{\chaptitlefont}{% Overwrites toc
    \normalfont\Huge\bfseries\sffamily\raggedleft\color{CinBlueText}}
}

\addtodef{\printpartname}{\color{CinBlueText}}{}% Part
\addtodef{\printchaptername}{\color{CinBlueText}}{}% Chapter
\setsecheadstyle{\Large\bfseries\color{CinBlueText}}% Section
\setsubsecheadstyle{\large\bfseries\color{CinBlueText}}% SubSection
\setsubsubsecheadstyle{\normalfont\bfseries\color{CinBlueText}}% SubSubSection
\setparaheadstyle{\normalfont\bfseries\color{CinBlueText}}% Paragraph
\setsubparaheadstyle{\normalfont\bfseries\color{CinBlueText}}% SubParagraph

% Table of contents
\addtodef{\tocheadstart}{\color{CinBlueText}}{} % If you want the whole TOC to be blue also
%\addtoiargdef{\printtoctitle}{\color{CinBlueText}}{} % If you just want the TOC title blue

% PDF properties
\hypersetup{colorlinks=true,
  linkcolor=[named]{CinBlueText},
  citecolor=[named]{CinBlueText},
  filecolor=[named]{CinBlueText},
  urlcolor=[named]{CinBlueText},
  bookmarksnumbered=true,
  pdftitle={The Comprehensive User Manual},
  pdfauthor={The Cinelerra-GG Community},
  pdfsubject={Video Editing},
  pdfkeywords={Cinelerra-GG, CGG, Cin5, Infinity, User Manual, Video
    editing system, Video editing program}
}

% %%%%%%%%%%%%%%%%%%%%%%%%%%%%%%%%%%%%%%%%%%%%%%%%%%%%%%%%%%%%%%%%%%%%

% Package listings
\lstset{                            % begin settings
  %language=R,                      % the language of the code
  inputencoding=utf8,
  basicstyle=\ttfamily\footnotesize,% the size of the fonts that are used for the code
  numbers=left,                     % where to put the line-numbers
  numberstyle=\tiny\color{black},   % the style that is used for the line-numbers
  stepnumber=1,                     % the step between two line-numbers. If it's 1, each line
                                    % will be numbered
  numbersep=5pt,                    % how far the line-numbers are from the code
  %backgroundcolor=\color{white},   % choose the background color. You must add \usepackage{color}
  showspaces=false,                 % show spaces adding particular underscores
  showstringspaces=false,           % underline spaces within strings
  showtabs=false,                   % show tabs within strings adding particular underscores
  frame=lines,                      % adds a frame around the code
  %frame=single,                    % adds a frame around the code
  rulecolor=\color{CinSilver},      % if not set, the frame-color
                                    % may be changed on line-breaks
                                    % within not-black text
                                    % (e.g. commens (green here)).
  tabsize=2,                        % sets default tabsize to 2 spaces
  captionpos=b,                     % sets the caption-position to bottom
  breaklines=true,                  % sets automatic line breaking
  breakatwhitespace=false,          % sets if automatic breaks should only happen at whitespace
  title=\lstname,                   % show the filename of files included with \lstinputlisting;
                                    % also try caption instead of title
  keywordstyle=\color{CinGreen},    % keyword style
  commentstyle=\color{gray},        % comment style
  stringstyle=\color{black},        % string literal style
  %backgroundcolor=\color{green!10},
  escapeinside={\%*}{*)},% FIXME?           % if you want to add a comment within your code
  extendedchars=true,
  %keepspaces = true                %!!!! spaces in comments
  texcl=true,
  postbreak=\mbox{\textcolor{CinSilver}{$\hookrightarrow$}\space},
  % morekeywords={*,...}% FIXME              % if you want to add more keywords to the set
}


%====================== Page geometry
% \geometry{left=2.0cm}
% \geometry{right=2.0cm}
% \geometry{top=2.0cm}
% \geometry{bottom=2.0cm}

\parindent=0.0cm                    % first indent in section
\righthyphenmin=2                   % hyphen last charecter
\setsecnumdepth{subsubsection} 		% section numeration depth

%\pagestyle{plain}
%\pagenumbering{roman}
%\renewcommand{\chapterheadstart}{
%%\vspace*{\beforechapskip}
%\hrule\medskip}
%\renewcommand{\chapnamefont}{\normalfont\large\scshape}
%\renewcommand{\chapnumfont}{\normalfont\large\scshape}
%\renewcommand{\chaptitlefont}{\normalfont\large\scshape}
%\renewcommand{\printchaptername}{\normalfont\large\scshape История}
%\renewcommand{\chapternamenum}{ }
%\renewcommand{\printchapternum}{\chapnumfont \thechapter}
%\renewcommand{\afterchapternum}{. }
%\renewcommand{\afterchapskip}{\vspace{2ex}}
%\renewcommand{\afterchaptertitle}{\par\nobreak\medskip\hrule\vskip 
%\afterchapskip}
%}
%\chapterstyle{madsen}		% one of chapter header style for memoir documentclass
\chapterstyle{morrow}		% one of chapter header style for memoir documentclass
% \renewcommand{\printchaptername}{\normalfont\large\scshape Chapter}
\renewcommand{\chapterheadstart}{}
%\renewcommand{\beforechapskip}{\vspace{2pt}}

%\renewcommand{\@pnumwidth}{3em} % memoir class, more space between chapter number and text.
\setlength{\cftfigurenumwidth}{4em} % memoir class, more space between figure number and text.

\renewcommand{\nomname}{Glossary} % glossary name

% Define the Cinnelerra-GG wordmark - a rough draft that can be
% adapted at any time. To be used in the text with \CGG{}
% or % \CGGI{}
\def\GG{\textsc{G\kern-0.1em G}}
\def\CGG{\textsc{Cinelerra-\GG}}
\def\INF{\textsc{Infinity}}
\def\CGGI{\CGG\;\INF}

% Hyphenation for unknown words and technical terms
\hyphenation{
  plug-ins
  ex-pan-ders
}


%%% Local Variables:
%%% mode: latex
%%% TeX-master: "../CinelerraGG_Manual"
%%% End:
