% FIXME Settings
% - Page numbers always on the right, matching the chapter
%   number.

% FIXME Text
% - Change all tables to variable dimensions. -> use draft
% - Remove all leading \$ characters from the shell examples.

% The Settings

% Set up the page layout
% memman.pdf capter 2: Laying out the page (READ IT!)
% example: memsty.sty
\settypeblocksize{8.75in}{34pc}{*}

% In loose reference to the theme CW.
\definecolor{CinBlueText}{RGB}{23,85,142}% "Dark blue"
\definecolor{CinBlue}{RGB}{35,134,220}%    "Light blue"
\definecolor{CinRed}{RGB}{205,38,11}%      "negativ"
\definecolor{CinOrange}{RGB}{250,125,0}%   "neutral"
\definecolor{CinGreen}{RGB}{39,174,96}%    "positiv"
\definecolor{CinSilver}{RGB}{127,140,141}%
\definecolor{CinWhite}{RGB}{239,240,241}%
\definecolor{CinDarkGray}{RGB}{35,38,41}%

% Original
% \definecolor{chaptercolour}{RGB}{23,85,142}
% \definecolor{sectioncolour}{RGB}{23,85,142}
% \definecolor{subsectioncolour}{RGB}{23,85,142}
% \definecolor{subsubsectioncolour}{RGB}{23,85,142}

\makechapterstyle{morrow}{% requires graphicx package
  \chapterstyle{default}
  \renewcommand*{\chapnamefont}{%
    \normalfont\Large\scshape\raggedleft\color{CinBlueText}}
  \renewcommand*{\chapnumfont}{%
    \normalfont\Large\bfseries\sffamily\color{CinBlueText}}
  \renewcommand*{\printchapternum}{%
    \chapnumfont\resizebox{!}{3ex}{\thechapter}}
  \renewcommand*{\afterchapternum}{% FIXME vskip?
    \par\hspace{1.5cm}\hrule\vskip\midchapskip}
  \renewcommand*{\chaptitlefont}{% Overwrites toc
    \normalfont\Huge\bfseries\sffamily\raggedleft\color{CinBlueText}}
}

\addtodef{\printpartname}{\color{CinBlueText}}{}% Part
\addtodef{\printchaptername}{\color{CinBlueText}}{}% Chapter
\setsecheadstyle{\Large\bfseries\color{CinBlueText}}% Section
\setsubsecheadstyle{\large\bfseries\color{CinBlueText}}% SubSection
\setsubsubsecheadstyle{\normalfont\bfseries\color{CinBlueText}}% SubSubSection
\setparaheadstyle{\normalfont\bfseries\color{CinBlueText}}% Paragraph
\setsubparaheadstyle{\normalfont\bfseries\color{CinBlueText}}% SubParagraph

% Table of contents
\addtoiargdef{\printtoctitle}{\color{CinBlueText}}{} % If you just want the TOC title blue
% \addtodef{\tocheadstart}{\color{CinBlueText}}{} % If you want the
% whole TOC to be blue also. The page numbers are also colored, but
% these are not links, that's confusing.


% PDF properties
\hypersetup{colorlinks=true,
  linkcolor=[named]{CinBlueText},
  citecolor=[named]{CinBlueText},
  filecolor=[named]{CinBlueText},
  urlcolor=[named]{CinBlueText},
  bookmarksnumbered=true,
  pdftitle={Cinelerra-GG -- The Comprehensive User Manual},
  pdfauthor={The Cinelerra-GG Community},
  pdfsubject={Video Editing},
  pdfkeywords={Cinelerra-GG, CGG, Cin5, Infinity, User Manual, Video
    editing system, Video editing program}
}


% Package listings
\lstset{% Common settings
  frame=single,
  framerule=0pt,
  framextopmargin=0.25ex,
  framexbottommargin=0.25ex,
  backgroundcolor=\color{CinWhite},
  basicstyle=\small,
  % No numbers by default. If required, activate explicitly in the
  % respective lstlisting: numbers=left|right or define a new style
  % below.
  numbers=none,
  numberstyle=\small\color{CinSilver},
  numbersep=1em,   % how far the line-numbers are from the code
  % Do not show:
  showspaces=false,
  showstringspaces=false,
  showtabs=false,
  %
  tabsize=2,
  breaklines=true, % sets automatic line breaking
  % Still undecided:
  title=\lstname,  % show the filename of files included with
                   % \lstinputlisting; also try caption instead of
                   % title
  inputencoding=utf8,
  extendedchars=true,
  % Required for c&p (PDF -> Clipboard):
  columns=fullflexible,
  postbreak=\mbox{\textcolor{CinSilver}{$\hookrightarrow$}\space},
  keywordstyle=\color{CinDarkGray},
  commentstyle=\footnotesize\color{gray},
  stringstyle=\color{CinDarkGray},
  % Retains the characters as entered:
  literate={-}{-}1 {*}{*}1
}
% Our settings specifically for the shell.
% Usage: \begin{lstlisting}[style=sh]
\lstdefinestyle{sh}{%
  language=bash,
  morekeywords={cp,gdb,git,grep,make,mkdir,tee}
}
% A pseudo-style that does nothing.
% Usage: \begin{lstlisting}[style=nil] instead of:
%        \begin{lstlisting}[]
\lstdefinestyle{nil}{}


% %%%%%%%%%%%%%%%%%%%%%%%%%%%%%%%%%%%%%%%%%%%%%%%%%%%%%%%%%%%%%%%%%%%%

\parindent=0.0cm                    % first indent in section
\righthyphenmin=2                   % hyphen last charecter
\setsecnumdepth{subsubsection} 		% section numeration depth

% We use the chapter header style:
\chapterstyle{morrow}

% \renewcommand{\printchaptername}{\normalfont\large\scshape Chapter}
\renewcommand{\chapterheadstart}{}
%\renewcommand{\beforechapskip}{\vspace{2pt}}

%\renewcommand{\@pnumwidth}{3em} % memoir class, more space between chapter number and text.
\setlength{\cftfigurenumwidth}{4em} % memoir class, more space between figure number and text.

\renewcommand{\nomname}{Glossary} % glossary name

% %%%%%%%%%%%%%%%%%%%%%%%%%%%%%%%%%%%%%%%%%%%%%%%%%%%%%%%%%%%%%%%%%%%%

% Define the Cinnelerra-GG wordmark - a rough draft that can be
% adapted at any time. To be used in the text with \CGG{}
% or % \CGGI{}
\def\GG{\textsc{G\kern-0.1em G}}
\def\CGG{\textsc{Cinelerra-\GG}}
\def\INF{\textsc{Infinity}}
\def\CGGI{\CGG\;\INF}

% Hyphenation for unknown words and technical terms
\hyphenation{%
  plug-ins
  ex-pan-ders
}


%%% Local Variables:
%%% mode: latex
%%% TeX-master: "../CinelerraGG_Manual"
%%% End:
